%<><><><><><><><><><><><>
% Introduction
%<><><><><><><><><><><><>

In 1888 the Austrian botanist Friedrich Reinitzer remarked the curious melting behaviour of cholesteryl benzoate \cite{reinitzer1888} (English translation: \cite{reinitzer1888eng}). The chemical displayed two distinct melting points: first melting into a cloudy liquid, then upon further heating, melting again into a clear liquid. 
Reinitzer promptly corresponded his findings, along with samples of cholesterol benzoate and cholesterol acetate, to physicist Otto Lehmann in March that same year \cite{knoll2010otto}. Lehmann looked more closely at these samples and discovered the crystalline order present in the liquid. Publishing his findings in 1889 \cite{lehmann1889}, liquid crystal (LC) research began in earnest.

As the name implies, liquid crystals are unique in their properties that combine both those of liquids (such as flow, inability to support shear, and the formation/coallescence of droplets) and those of crystals (such as anisotropy of electromagnetic/optical properties and the periodic arrangement of molecules in spatial dimensions) \cite{lcintro}. 



LCs in nanoscience: controlled anisotropy of carbon nanotubes dispersed in liquid crystal
The nanosciences have also picked up on the potential of LCs. Typically, it is desirable for carbon nanotubes (CNTs) to have a certain degree of alignment for their applications. By dispersion in a LC solvent, CNTs can be made to align with the LC nematic director \cite{lynch2002,dierking2004,cnt_lc}.

LCs have also attracted interest in self-assembly nanomaterials research. Ref.\ \cite{selfassembly}, simulating a nanoparticle and LC mixture under a variety of quenches and concentrations, found that nanoparticles could be concentrated in defects or in the channels and pockets formed by slower growing regions of the nematic-isotropic interface as nematic regions expanded. Directing the assembly of nanopatricles into addressable arrangements can lead to materials with high processability, self-healing properties, and reversible control \cite{selfassembly}.

``A new era for liquid crystal research: applications of liquid crystals in soft matter nano-, bio-and microtechnology'' \cite{LCreview}

Interesting emergent technology: liquid crystal elastomers offer interesting temperature dependent reactions such as dramatic volumetric changes or a transparency \cite{lcelastomers}.

Liquid crystal phases have been found both \textit{in vivo} as well as \textit{in vitro}  for major classes of biological compounds including lipids, proteins, carbohydrates and nucleic acids \cite{hamley2010}. Nematic and chiral nematic (cholesteric) phases are most common in this domain, but hexagonal columnar (such as in DNA) and smectic phases (such as in the arrangement of amylopectin side chains in starch) have also been observed \cite{hamley2010}. 

In the production of spider dragline silk, an aqueous solution of the silk fibroin takes on a nematic phase as an orienting mechamisn at an intermediate stage of the process \cite{spidersilk1, spidersilk2, spidersilk3}. 
Cases of LC mesophases have been observed in certain DNA packings: inside the heads of bacteriophages \cite{earnshaw1980dna}, dinoflagellates \cite{livolant1978}, or plasmid DNA within bacteria \cite{reich1994liquid}.
Inside bacteriophages, the packing takes on a columnar shape within concentric rings \cite{cerritelli1997encapsidated}. In dinoflagellates, the chromosomes assume a twisted cholesteric phase. And at physiological concentrations, \textit{in vitro} plasmid DNA of \textit{Escherichia coli} bacteria show birefringent LC textures of a cholesteric phase \cite{reich1994liquid}.

Nematic and chiral nematic (cholesteric) phases are the most common in such systems, but hexagonal columnar (such as in ) \cite{hamley2010}


In \cite{lewis2014} they find nice results imaging a couple viruses in these confinements



